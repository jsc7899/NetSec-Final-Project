%%%%%%%%%%%%%%%%%%%%%%%%%%%%%%%%%%%%%%%%%
% a0poster Landscape Poster
% LaTeX Template
% Version 1.0 (22/06/13)
%
% The a0poster class was created by:
% Gerlinde Kettl and Matthias Weiser (tex@kettl.de)
% 
% This template has been downloaded from:
% http://www.LaTeXTemplates.com
%
% License:
% CC BY-NC-SA 3.0 (http://creativecommons.org/licenses/by-nc-sa/3.0/)
%
%%%%%%%%%%%%%%%%%%%%%%%%%%%%%%%%%%%%%%%%%

%----------------------------------------------------------------------------------------
%	PACKAGES AND OTHER DOCUMENT CONFIGURATIONS
%----------------------------------------------------------------------------------------

\documentclass[a0,landscape]{a0poster}

\usepackage{multicol} % This is so we can have multiple columns of text side-by-side
\columnsep=100pt % This is the amount of white space between the columns in the poster
\columnseprule=3pt % This is the thickness of the black line between the columns in the poster

\usepackage[svgnames]{xcolor} % Specify colors by their 'svgnames', for a full list of all colors available see here: http://www.latextemplates.com/svgnames-colors

\usepackage{times} % Use the times font
%\usepackage{palatino} % Uncomment to use the Palatino font

\usepackage{graphicx} % Required for including images
\graphicspath{{figures/}} % Location of the graphics files
\usepackage{booktabs} % Top and bottom rules for table
\usepackage[font=small,labelfont=bf]{caption} % Required for specifying captions to tables and figures
\usepackage{amsfonts, amsmath, amsthm, amssymb} % For math fonts, symbols and environments
\usepackage{wrapfig} % Allows wrapping text around tables and figures
 \usepackage{float}
 \usepackage{hyperref}
\begin{document}

%----------------------------------------------------------------------------------------
%	POSTER HEADER 
%----------------------------------------------------------------------------------------

% The header is divided into three boxes:
% The first is 55% wide and houses the title, subtitle, names and university/organization
% The second is 25% wide and houses contact information
% The third is 19% wide and houses a logo for your university/organization or a photo of you
% The widths of these boxes can be easily edited to accommodate your content as you see fit

\begin{minipage}[b]{0.55\linewidth}
\veryHuge \color{NavyBlue} \textbf{Honeypot Configuration and Data Analysis} \color{Black}\\ % Title
%\Huge\textit{Honeypot Configuration and Data Analysis}\\[1cm] % Subtitle
\huge \textbf{Jared Campbell \& David Zehden}\\ % Author(s)
\huge The University of Texas at Austin\\ % University/organization
\end{minipage}
%
%
%
\begin{minipage}[b]{0.19\linewidth}
%\includegraphics[width=20cm]{logo.png} % Logo or a photo of you, adjust its dimensions here
\end{minipage}

\vspace{1cm} % A bit of extra whitespace between the header and poster content

%----------------------------------------------------------------------------------------

\begin{multicols}{3} % This is how many columns your poster will be broken into, a poster with many figures may benefit from less columns whereas a text-heavy poster benefits from more

%----------------------------------------------------------------------------------------
%	ABSTRACT
%----------------------------------------------------------------------------------------

\color{Black} % Navy color for the abstract

%\begin{figure}[H]
%	\begin{center}
%	\includegraphics[width=30cm]{mhn.png}
%	\caption{The Modern Honeypot Network Dashboard}
%	\end{center}
%\end{figure} 
%\begin{center}[H]
%	\includegraphics[width=20cm]{mhn.png} % Logo or a photo of you, adjust its dimensions here
	
%\end{center}
%----------------------------------------------------------------------------------------
%	OBJECTIVES
%----------------------------------------------------------------------------------------

\color{Black} % DarkSlateGray color for the rest of the content

\section*{Main Objectives}

\begin{enumerate}
\item Setup Amazon Web Services EC2 instance
\item Explore Honeypot options
\item Deploy Honeypot of choice on AWS
\item Configure Honeypot as needed
\item Collect data over time
\item Explore collected data to determine trends
\end{enumerate}

%----------------------------------------------------------------------------------------
%	MATERIALS AND METHODS
%----------------------------------------------------------------------------------------

\section*{Results}
Our honeypot made over 20,000 detections. We received traffic from all over the world, as you can see in Figure 2. In order to generate this map, we took all the IPs we detected and used the \href{https://github.com/pieqq/PyGeoIpMap}{PyGeoIpMap} library.
\begin{figure}[H]
	\begin{center}
		\includegraphics[width=30cm]{map.png}
		\caption{Approximate Locations of Detected IP Addresses}
	\end{center}
\end{figure} 

\begin{figure}[H]
	\begin{center}
		\includegraphics[width=25cm]{top_countries.png}
		\caption{Breakdown of Detections by Country}
	\end{center}
\end{figure} 

\begin{figure}[H]
	\begin{center}
		\includegraphics[width=20cm]{time_breakdown.png}
		\caption{Breakdown of Detections by Time of Day (UTC)}
	\end{center}
\end{figure} 

\begin{figure}[H]
	\begin{center}
		\includegraphics[width=25cm]{top_os.png}
		\caption{Breakdown of Detected Operating Systems}
	\end{center}
\end{figure} 

\begin{figure}[H]
	\begin{center}
		\includegraphics[width=20cm]{images/top_attackers.png}
		\caption{IP Addresses of Top Attackers}
	\end{center}
\end{figure} 

\begin{figure}[H]
	\begin{center}
		\includegraphics[width=20cm]{images/top_users.png}
		\caption{Top Usernames Attempted}
	\end{center}
\end{figure}

\begin{figure}[H]
	\begin{center}
		\includegraphics[width=20cm]{images/top_pass.png}
		\caption{Top Passwords Attempted}
	\end{center}
\end{figure}


%------------------------------------------------
%----------------------------------------------------------------------------------------
%	RESULTS 
%----------------------------------------------------------------------------------------

\section*{Conclusion}
In order to research the effectiveness of honeypots, we implemented and configured our own honeypot on an AWS server. Our honeypot network collected even more data than we expected over the course of a single week. We have analyzed the data collected by our honeypot network to find trends in the origin of attacks, the hardware used for attacks, which services are most heavily targeted, and more. 

We have also seen that different honeypot solutions offer both various types of data and different quantities of data. From our findings, we could see that Snort was was able to capture a massive volume of information, but it was data that was generally more surface level. In contrast, Dionaea could provide examples of malware binaries that attackers attempted to run. However, this happened very infrequently, so it ultimately did not provide a great deal of data.

Our research has demonstrated that honeypots are an effective tool for monitoring malicious activity on a network. Our research also shows that honeypots can collect large amounts of data on many different types of trends. The data collected by a honeypot can be tailored to suit any organizations specific needs, making honeypots an effective tool not only for research, but also for securing enterprise environments. 

%----------------------------------------------------------------------------------------

\end{multicols}
\end{document}
