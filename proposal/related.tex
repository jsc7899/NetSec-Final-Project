% Try to mimic how the papers we have read in class use citations and references.
%
Neil Fox's post about configuring Dionaea and Cowrie was very useful in learning how to setup the tools in our honeypot ~\cite{Fox}. In his post he describes how to install, configure, and add improvements to Dionaea and Cowrie. He recommends implementing log rotation to prevent the bistreams logs that Dionaea uses from filling up. He also implements additional improvements to Cowrie that allow an administrator to replay the BASH sessions of an attacker once they are inside the honeypot. We intend to implement both log rotation and replay in our own design.

Rapid7 wrote a guide for configuring honeypots on AWS ~\cite{Rapid7}. Although the guide is specific to Rapid7's InsightIDR honeypots, the configuration of their AWS environment was particularly valuable. They describe in detail how to launch an EC2 instance, create security groups, and create IAM roles. We referred to this article for our own configuration of our AWS environment. 

Steve Gathof also wrote an excellent article on the setup of a honeypot on an AWS EC2 instance ~\cite{Gathof}. While his work will likely be a useful reference for our architecture, it is not explicitly related to the analysis that we will be performing. 

Additionally, Polyakov et al. explored both the architecture and several specific functions that a honeypot should fulfill ~\cite{Polyakov2018ArchitectureOT}. We will try to implement the functions mentioned in the Polyakov's article, in particular, creating variable conditions for accessing the system in order to possibly filter out low level intruders.

Kuwatly et al. discuss the design of a dynamic honeypot, which is "an autonomous honeypot capable of adapting in a dynamic and constantly changing network environment" ~\cite{Dynamic}. In their paper, they describe how to combine active probing, passive fingerprinting, and a network intrusion detection system with a honeypot. This allows for the dynamic honeypot to automatically update and adjust to changes in the network environment in which it is located. While our network environment  is currently comprised of only a single server, we would like to eventually extend our work to a larger network of multiple honeypots. If we are able to extend our work as such, this paper will be an invaluable reference. 