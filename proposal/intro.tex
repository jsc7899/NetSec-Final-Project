A honeypot is a server that is made intentionally vulnerable in order to attract the attention of malicious actors. The server then logs any attempts by attackers to exploit and gain access to it. The goal is for researchers to be able to analyze common trends in the kinds of attack used by attackers and even possibly discover new kinds of attacks that have never been seen before. Another use case for honeypots in an enterprise environment is to slow down attackers by allowing them to attack the non-critical honeypot instead critical infrastructure. 

There are many different tools that can be used to create a honeypot. For example, the honeypot could emulate a vulnerable web app or a vulnerable end-user machine. The configuration of the honeypot depends on the goals of its creators whether that be research or protection as described above. Our goal is to research these various tools and gather enough data to be able provide an analysis of current trends in attacks.

For our approach, we will configure a honeypot as a vulnerable server using various tools which we have found to emulate common vulnerabilities and log attack data. The key insight of our project is dependent on the data which collect as there a numerous outcomes depending on the types of attacks we receive. 
