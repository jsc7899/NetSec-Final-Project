A honeypot is a server that is made intentionally vulnerable in order to attract the attention of malicious actors. The server then logs any attempts by attackers to exploit and gain access to it. The goal is for researchers to be able to analyze common trends in the kinds of attacks used by malicious actors and even possibly discover new kinds of attacks that have never been seen before. Another use case for honeypots in an enterprise environment is to slow down attackers by allowing them to attack the non-critical honeypot instead critical infrastructure. 

There are many different tools that can be used to create a honeypot. For example, the honeypot could emulate a vulnerable web app or a vulnerable end-user machine. The configuration of the honeypot depends on the goals of its creators, whether that be research or protection as described above. Our goal is to research these various tools and configurations, and to gather enough data to be able provide an analysis of current trends in attacks. Since there are so many different tools and honeypot configurations, our implementation will focus on a general honeypot meant to collect as much data as possible. The results of our research will demonstrate the effectiveness of a honeypot in collecting different types of data which could be used to improve the security of an organization and provide knowledge on general trends to the information security community. 

For our approach, we have configured a honeypot as a vulnerable server using various tools which we have found to emulate common vulnerabilities and log attack data. We have also configured a front end website for the honeypot. The collected data is automatically analyzed and visualized via Python scripts, and the results are displayed on the website. The most interesting part of our design is that once the honeypot is configured, it runs continually without any need for further user interaction. The longer the honeypot runs, the more data it collects, and the more accurate trends over time become. As we were only able to run our fully configured honeypot for approximately one week, the analysis only shows common trends over a one week time period. However, given more time, the honeypot could eventually demonstrate trends over months or even years. The longer the honeypot is active, the more interesting the data becomes. 

In our initial results, we were able to track the top attacker IP addresses and their country of origin, attacker operating systems, targeted ports, timing of attacks, and SSH attack data such as the top usernames and passwords of all login attempts. Interestingly, the top five countries from which attacks originated were the United States, Netherlands, Russia, China, and Romania (in order of number of detections). The two most widely used operating systems by attackers were Linux 2.2.x-3.x and Windows 7 or 8 (respectively). It is interesting that attackers are using fairly outdated operating systems (current Linux kernel is 4.14 and Windows 7 will reach end of life in January 2020). 

In the future, we would like to continue to run the honeypot server in order to collect data over a larger time period. It will be interesting to see how trends change across weeks and months as well as overall long-term trends. There is also much more we could configure on the honeypot, such as making masking certain elements of the honeypot detection programs to make the honeypot seem like a more realistic target. We theorize that the more enticing the honeypot looks to attackers, the more likely they would be to attack it and possibly even use more advanced attacks, all of which would provide us with even more detailed data. 