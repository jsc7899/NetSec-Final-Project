The three sensors used by the honeypot, Dionaea, Cowrie, and Snort, were each initially individually configured through MHN. We then manually configured the logging levels of each sensors to ensure relevant data was logged separately from error and debugging information so that it could be parsed by our Python analysis scripts. 

We have also configured the AWS server's firewall and security groups. We are allowing all inbound network traffic on all ports monitored by the honeypot (SSH, FTP, HTTP(S), etc.) and denying all outbound traffic. SSH access is restricted to our own SSH keys, and the SSH port has been changed from the default of 22 to allow Cowrie to monitor attacks on port 22. Password logins are denied and root login is also denied. 

For data analysis, we used the \texttt{mongoexport} command line tool to export data collected by MHN to JSON files. We used \texttt{SCP} to securely copy these files onto our local machines. Then we used Jupyter notebooks to explore data. Primarily, the \texttt{matplotlib}, \texttt{numpy}, and \texttt{pandas} libraries were used to aid the development of our findings. Additionally, we modified MHN to include a new webpage for particularly interesting findings. Such findings were generated from a daily \texttt{cron} job we set up on the server to run scripts that we developed locally.