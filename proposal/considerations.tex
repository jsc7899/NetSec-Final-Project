Since we are using AWS to host our honeypot, we must consider the configuration of the AWS instance itself. In particular, we will consider AWS firewall rules and security groups to allow all incoming connections and deny all outgoing connections. We must also consider that AWS might shutdown our server if they detect it is under attack. In that case, we will have a backup of the server prepared and it will be migrated to a private server which we control. Finally, if our honeypot does not collect enough attack data in the time period during which it is active, we will attack the server ourselves to demonstrate its capabilities. 

If time permits, we have further plans to add more AWS honeypot servers and network them together with our original instance to hopefully see how an attack might traverse the network. We can also implement an intrusion detection system such as \href{https://www.snort.org/}{Snort}if we have enough time. 