The honeypot is hosted on an Amazon Web Services (AWS) virtual server instance. We have had to upgrade our instance to increase it's storage and memory. The server now runs an EC2 t3a.medium instance with two processor cores, 4GB or RAM, and 30GB of SSD storage. The server runs an Ubuntu Server 16 image as its operating system. 

To create the functionality of the honeypot, we have configured several tools which act as sensors or detectors. \href{https://github.com/DinoTools/dionaea}{Dionaea} captures malware and can simulate certain individual vulnerabilities such as SQL, FTP, SMB, and HTTP vulnerabilities. \href{https://github.com/cowrie/cowrie}{Cowrie} is an SSH honeypot which logs brute force attacks and shell interaction. \href{https://www.snort.org/}{Snort} is a network intrusion detection system which does not simulate any vulnerabilities itself, but detects and logs attacks targeted at the vulnerabilities provided by other tools. These sensors are all managed by the \href{https://github.com/pwnlandia/mhn}{Modern Honeypot Network (MHN)}. MHN provides a single platform for managing honeypots and offers several useful analytical tools for assessing high level attack metrics as well as management tools to easily modify, add, and remove honeypot instances. 

The front-end website which we use to monitor our honeypot and its sensors, as well as visualize the data they collect, was initially provided by MHN. We have modified the MHN site to fit our needs by configuring it specifically for our sensors, and adding additional visualizations and analyses. 

These visualization and analyses were created using Python to graph trends. There are several Python scripts which run on the server to continually analyze the data collected by the honeypot and construct graphs and charts which are then displayed on the main front-end site. 