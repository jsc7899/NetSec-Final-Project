We have set up a basic Amazon Web Services server. The server is an EC2 t2.micro instance. We are using an Ubuntu Server 18 image as our operating system. 

To create the functionality of the honeypot, we have found several tools which we will configure and test. \href{https://github.com/DinoTools/dionaea}{Dionaea} captures malware and can simulate certain individual vulnerabilities. \href{https://github.com/cowrie/cowrie}{Cowrie} is an SSH honeypot which logs brute force attacks and shell interaction. In order to manage these honeypot technologies we will use \href{https://github.com/pwnlandia/mhn}{Modern Honeypot Network (MHN)}. MHN provides a single platform for managing honeypots and offers several useful analytics tools for assessing high level attack metrics as well as management tools to easily modify, add, and remove honeypot instances.

For data analysis we will mainly use Python to graph trends. Additionally, MHN provides some nice data visualizations of attack data it receives from connected honeypots. \href{https://github.com/mushorg/tanner/}{Tanner} is another tool we can test which analyzes data specifically from SNARE, mentioned above. 
