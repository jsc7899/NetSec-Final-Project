\documentclass{sig-alternate}
%\documentclass[11pt,letterpaper,singlecolumn]{article}
%\usepackage{url}
\usepackage{epsfig}
%\usepackage{subfigure}
%%\usepackage{latexsym}
%%\usepackage{times}
%%\usepackage{fancyhdr}
%%\usepackage{multirow}
%\usepackage{listing}
\usepackage{soul}
\usepackage{algorithmic}
\usepackage{algorithm}
\usepackage{hyperref}

\newcommand{\parbold}[1]{\noindent{\bf #1}}
%%%%%%%%%%%%%%%%%%%%%%%%%%
%%% Remarks
\newif\ifremark
\long\def\remark#1{
\ifremark%
        \begingroup%
        \dimen0=\columnwidth
        \advance\dimen0 by -1in%
        \setbox0=\hbox{\parbox[b]{\dimen0}{\protect\em #1}}
        \dimen1=\ht0\advance\dimen1 by 2pt%
        \dimen2=\dp0\advance\dimen2 by 2pt%
        \vskip 0.25pt%
        \hbox to \columnwidth{%
                \vrule height\dimen1 width 3pt depth\dimen2%
                \hss\copy0\hss%
                \vrule height\dimen1 width 3pt depth\dimen2%
        }%
        \endgroup%
\fi}

%%%%%%%%%%%%%%%%%%%%%%%%%%
%%% Block comments
\newcommand{\ignore}[1]{}

%%%%%%%%%%%%%%%%%%%%%%%%%%%%%%%%%%%%%
\remarkfalse
%\remarkfalse
% \remark{this is a comment that shows up in text
% Switch remarkfalse on to turn comments off } 
% -- use in body, not up here
%%%%%%%%%%%%%%%%%%%%%%%%%%%%%%%%%%%%%


\begin{document}


\title{ Building a Honeypot Server \\And Analyzing the Data it Collects}
\author{Jared Campbell\\The University of Texas at Austin  \and David Zehden\\The University of Texas at Austin}

\pagenumbering{arabic}
%\date{}  % comment this out if you want the date to print

\maketitle

\begin{abstract}
This project aims to build a honeypot server and analyze the data it collects. The goal of a honeypot server is to create vulnerable server on the open internet which we expect will be attacked by malicious actors. These attacks will be logged by the server, then we will analyze the data collected to find common trends in the attacks. Our honeypot will collect data on both an attacking machine and the attacks it directs at the honeypot. Our honeypot server is hosted on an Amazon Web Services (AWS) virtual server instance. The network for the server has been configured to allow all incoming traffic on monitored ports we expect to be targeted (such as SSH). We then analyzed the collected data using Python to find trends in attacker IP addresses, attacker operating systems, attacked ports, and timing of attacks. The honey pot was successful in collecting a large amount of attack data, averaging over 1000 unique data points per day. Our research has determined that a honeypot server is an effective tool for monitoring potential attacks on a network. We have also found honeypots to be highly configurable which allows for the collection of data specific to an organization's needs. 
 
\end{abstract}

\section{Introduction}
\label{sec:intro}
A honeypot is a server that is made intentionally vulnerable in order to attract the attention of malicious actors. The server then logs any attempts by attackers to exploit and gain access to it. The goal is for researchers to be able to analyze common trends in the kinds of attacks used by malicious actors and even possibly discover new kinds of attacks that have never been seen before. Another use case for honeypots in an enterprise environment is to slow down attackers by allowing them to attack the non-critical honeypot instead critical infrastructure. 

There are many different tools that can be used to create a honeypot. For example, the honeypot could emulate a vulnerable web app or a vulnerable end-user machine. The configuration of the honeypot depends on the goals of its creators, whether that be research or protection as described above. Our goal is to research these various tools and configurations, and to gather enough data to be able provide an analysis of current trends in attacks. Since there are so many different tools and honeypot configurations, our implementation will focus on a general honeypot meant to collect as much data as possible. The results of our research will demonstrate the effectiveness of a honeypot in collecting different types of data which could be used to improve the security of an organization and provide knowledge on general trends to the information security community. 

For our approach, we have configured a honeypot as a vulnerable server using various tools which we have found to emulate common vulnerabilities and log attack data. We have also configured a front end website for the honeypot. The collected data is automatically analyzed and visualized via Python scripts, and the results are displayed on the website. The most interesting part of our design is that once the honeypot is configured, it runs continually without any need for further user interaction. The longer the honeypot runs, the more data it collects, and the more accurate trends over time become. As we were only able to run our fully configured honeypot for approximately one week, the analysis only shows common trends over a one week time period. However, given more time, the honeypot could eventually demonstrate trends over months or even years. The longer the honeypot is active, the more interesting the data becomes. 

In our initial results, we were able to track the top attacker IP addresses and their country of origin, attacker operating systems, targeted ports, and SSH attack data such as the top usernames and passwords of all login attempts. 

\section{Motivation}
\label{sec:motivation}
% Your motivation section should motivate your research. Your motivation reads more like background/literature review.
%

A great deal of the academic work regarding honeypots is focused on furthering some particular facet of a honeypot. While we certainly have incorporated ideas from such work, ours is primarily motivated by a desire to explore honeypots generally. The goals of our project are to collect as much data as possible and to explore the configuration of a honeypot network.

Additionally, we would like to gain a sense of the types of data we obtain and the high level information we can derive from such data. To this end, an exploration of the data we collect is necessary. This will allow us to understand the strengths and weaknesses of the various honeypot technologies we utilize. 

%	Mairh, et al. discuss a variety of use-cases for honeypots, among them is the use of a honeypot in conjunction with an Intruder Detection System (IDS). The authors note that an IDS typically uses misuse and anomaly detection [1]. We hope to both implement a system similar to the one described by Mairh, et al. and perform an analysis of the data we collect from the system in accordance with the type of anomaly detection described.

%Additionally, Chuvakin was able to classify individual types of attacks on his system [2]. Such a level of analysis is impressive, and we hope to achieve similar data collection, albeit in a lesser volume due to the significantly shorter time frame. Also, our analysis may be more focused on traffic metrics rather than the sort of in depth log analysis done by Chuvakin. We are more inclined towards determining if and when an attack has occurred rather than what specific attacks occurred.

% \ignore{comment text} is better than a line comment.
\ignore{Sometimes background is merged into motivation, and is not required separately.}

\section{Our Architecture}
\label{sec:arch}
The honeypot is hosted on an Amazon Web Services (AWS) virtual server instance. We have had to upgrade our instance to increase it's storage and memory. The server now runs an EC2 t3a.medium instance with two processor cores, 4GB or RAM, and 30GB of SSD storage. The server runs an Ubuntu Server 16 image as its operating system. 

To create the functionality of the honeypot, we have configured several tools which act as sensors or detectors. \href{https://github.com/DinoTools/dionaea}{Dionaea} captures malware and can simulate certain individual vulnerabilities such as SQL, FTP, SMB, and HTTP vulnerabilities. \href{https://github.com/cowrie/cowrie}{Cowrie} is an SSH honeypot which logs brute force attacks and shell interaction. \href{https://www.snort.org/}{Snort} is a network intrusion detection system which does not simulate any vulnerabilities itself, but detects and logs attacks targeted at the vulnerabilities provided by other tools. These sensors are all managed by the \href{https://github.com/pwnlandia/mhn}{Modern Honeypot Network (MHN)}. MHN provides a single platform for managing honeypots and offers several useful analytical tools for assessing high level attack metrics as well as management tools to easily modify, add, and remove honeypot instances. 

The front-end website which we use to monitor our honeypot and its sensors, as well as visualize the data they collect, was initially provided by MHN. We have modified the MHN site to fit our needs by configuring it specifically for our sensors, and adding additional visualizations and analyses. 

These visualization and analyses were created using Python to graph trends. There are several Python scripts which run on the server to continually analyze the data collected by the honeypot and construct graphs and charts which are then displayed on the main front-end site. 

\section{Setup and Configuration}
\label{sec:setup}
We have set up Dionaea on our AWS server and it is successfully collecting logs. Some further configuration may be required in order to ensure that we are logging at an appropriate level. We also plan to install and configure more modules for Dionaea including some for logging and some for adding more vulnerabilities. We also have Cowrie running on our AWS server. Modern Honeypot Network is running and provides up to date data collected by Dionaea and Cowrie, which already provides some high level insights. We may add additional sensors to MHN in  We have also configured the AWS server's firewall and security groups. Currently we are allowing all inbound network traffic on all ports and denying all outbound traffic. SSH access is restricted to our own SSH keys. Password logins are denied and root login is also denied. So far we have logged numerous attempts from scanners to log on to our service, and Dionaea has already captured a few malware binaries. 


\section{Results}
\label{sec:results}
Due to some technical issues, we had to reset our EC2 instance, thus the findings presented are those collected following the reset. As of the writing of this paper, we have received over 1,300 attacks on our honeypots. Modern Honeypot Network provides some useful high level statistics about our honeypots some of which may be seen in tables 1 and 2. Another interesting observation is that the vast majority of our data came from Snort, as over 1,200 attacks originated on our Snort honeypot.

We also found that 1433 was attacked significantly than any other port. This is likely due to port 1433's use as the default port for many SQL servers [TODO ADD CITATION].


\begin{table}[H]
	\resizebox{\columnwidth}{!}{%
	\begin{tabular}{|c|c|c|}
		\hline
		IP Address (truncated) & Country     & Number of attacks \\ \hline
		45.136                 & Germany     & 21                \\ \hline
		80.82.7                & Netherlands & 18                \\ \hline
		83.97.2                & Romania     & 17                \\ \hline
		89.248                 & Netherlands & 17                \\ \hline
		81.22.4                & Russia      & 16                \\ \hline
	\end{tabular}%
	}
\caption{Top Five Individual Detected IP Addresses} \label{tab:ips}
\end{table}

\begin{table}[H]
	\resizebox{\columnwidth}{!}{%
	\begin{tabular}{|c|c|c|}
		\hline
		Attacked Port & Number of Attacks & Common Port Use                                                                               \\ \hline
		1433          & 195               & SQL Server                                                                                    \\ \hline
		22            & 70                & SSH                                                                                           \\ \hline
		5060          & 68                & Clear Text SIP, VoIP                                                                          \\ \hline
		445           & 18                & Server Message Block                                                                          \\ \hline
		8545          & 17                & \begin{tabular}[c]{@{}c@{}}Remote Procedure Call interface\\ of Ethereum clients\end{tabular} \\ \hline
	\end{tabular}%
	}
\caption{Top Five Attacked Ports} \label{tab:ports}
\end{table}

Using the \href{https://github.com/pieqq/PyGeoIpMap}{PyGeoIpMap} library, we were also able to plot the approximate locations of attacker IP addresses (Figure 1). From this plot, we can see that attacks came from all over the world, though were particularly prominent from East/South East Asia, Brazil, and various parts of Europe.

\begin{figure}[H]
	\includegraphics[width=\linewidth]{output.png}
	\caption{Approximate Locations of Attacker IP Addresses.}
	\label{fig:map}
\end{figure}


\section{Considerations}
\label{sec:considerations}
Since we are using AWS to host our honeypot, we must consider the configuration of the AWS instance itself. In particular, we will consider AWS firewall rules and security groups to allow all incoming connections and deny all outgoing connections. We must also consider that AWS might shutdown our server if they detect it is under attack. In that case, we will have a backup of the server prepared and it will be migrated to a private server which we control. Finally, if our honeypot does not collect enough attack data in the time period during which it is active, we will attack the server ourselves to demonstrate its capabilities. 

If time permits, we have further plans to add more AWS honeypot servers and network them together with our original instance to hopefully see how an attack might traverse the network. We can also implement an intrusion detection system such as \href{https://www.snort.org/}{Snort}if we have enough time. 

\section{Related Work}
\label{sec:related}
Neil Fox's post about configuring Dionaea and Cowrie was very useful in learning how to setup the tools in our honeypot ~\cite{Fox}. In his post he describes how to install, configure, and add improvements to Dionaea and Cowrie. He recommends implementing log rotation to prevent the bistreams logs that Dionaea uses from filling up. He also implements additional improvements to Cowrie that allow an administrator to replay the BASH sessions of an attacker once they are inside the honeypot. We intend to implement both log rotation and replay in our own design.

Rapid7 wrote a guide for configuring honeypots on AWS ~\cite{Rapid7}. Although the guide is specific to Rapid7's InsightIDR honeypots, the configuration of their AWS environment was particularly valuable. They describe in detail how to launch an EC2 instance, create security groups, and create IAM roles. We referred to this article for our own configuration of our AWS environment. 

Steve Gathof also wrote an excellent article on the setup of a honeypot on an AWS EC2 instance ~\cite{Gathof}. While his work will likely be a useful reference for our architecture, it is not explicitly related to the analysis that we will be performing. 

Additionally, Polyakov et al. explored both the architecture and several specific functions that a honeypot should fulfill ~\cite{Polyakov2018ArchitectureOT}. We will try to implement the functions mentioned in the Polyakov's article, in particular, creating variable conditions for accessing the system in order to possibly filter out low level intruders.

Kuwatly et al. discuss the design of a dynamic honeypot, which is "an autonomous honeypot capable of adapting in a dynamic and constantly changing network environment" ~\cite{Dynamic}. In their paper, they describe how to combine active probing, passive fingerprinting, and a network intrusion detection system with a honeypot. This allows for the dynamic honeypot to automatically update and adjust to changes in the network environment in which it is located. While our network environment  is currently comprised of only a single server, we would like to eventually extend our work to a larger network of multiple honeypots. If we are able to extend our work as such, this paper will be an invaluable reference. 

\section{Conclusions}
\label{sec:conclusion}
In order to research the effectiveness of honeypots, we implemented and configured our own honeypot on an AWS server. Our honeypot network collected even more data than we expected over the course of a single week. We have analyzed the data collected by our honeypot network to find trends in the origin of attacks, the hardware used for attacks, which services are most heavily targeted, and more. 

We have also seen that different honeypot solutions offer both various types of data and different quantities of data. From our findings, we could see that Snort was was able to capture a massive volume of information, but it was data that was generally more surface level. In contrast, Dionaea could provide examples of malware binaries that attackers attempted to run. However, this happened very infrequently, so it ultimately did not provide a great deal of data.

Our research has demonstrated that honeypots are an effective tool for monitoring malicious activity on a network. Our research also shows that honeypots can collect large amounts of data on many different types of trends. The data collected by a honeypot can be tailored to suit any organizations specific needs, making honeypots an effective tool not only for research, but also for securing enterprise environments. 

%\section{References}


{ 
\bibliographystyle{abbrv}
\bibliography{biblio}
}
%[1] Mairh, Abhishek \& Barik, Debabrat \& Verma, Kanchan \& Jena, Debasish. (2011). Honeypot in network security: A survey. ACM International Conference Proceeding Series. 600-605. 10.1145/1947940.1948065. \linebreak

%[2] Chuvakin, A. (2003). “Honeynets: High Value Security Data”: Analysis of real attacks launched at a honeypot. Network Security, 2003(8), 11-15. \linebreak

%[3] Gathof, S. (2018). Deploying a Honeypot on AWS [Blog post]. Retrieved from \href{https://medium.com/@sudojune/deploying-a-honeypot-on-aws-5bb414753f32}{https://medium.com/
%	\linebreak@sudojune/deploying-a-honeypot-on-aws-5bb414753f32}\linebreak

%[4] Polyakov, V. V., \& Lapin, S. A. (2018, October). Architecture of the Honeypot System for Studying Targeted Attacks. In 2018 XIV International Scientific-Technical Conference on Actual Problems of Electronics Instrument Engineering (APEIE) (pp. 202-205). IEEE.

%[5] N. Fox, “Setting up dionaea \& cowrie with mhn,” Malware, Threat Hunting \& Incident Response, 24-Sep-2019. [Online]. Available: https://neil-fox.github.io/Setting-up-Dionaea-\&-Cowrie-with-MHN/. [Accessed: 04-Dec-2019].

\end {document}

